\section{Introduction}
\label{sec:introduction}

The Programming with Big Data in R~\citep{pbdr2012}, abbreviated pbdR or just pbd, is a project which seeks to elevate the \proglang{R} language to supercomputers.  This package, \pkg{pbdBASE}~\citep{Schmidt2012pbdBASEpackage}, contains a set of wrappers of the high performance libraries BLACS, PBLAS, and ScaLAPACK~\citep{slug}, and also a host of new subroutines for performing distributed matrix computations in \proglang{R}.  The package is a dependency of \pkg{pbdDMAT}~\citep{Schmidt2012pbdDMATpackage}, which is meant to greatly simplify the \pkg{pbdBASE} system into something that intimately resembles the \proglang{R} language.  Since these two packages ultimately rely on the ScaLAPACK library, the data type used with each is the block-cyclic distributed matrix.  See the \pkg{pbdDMAT} vignette for more details.

Updates and bug releases for this and other \pkg{pbd} projects may, especially while in infancy, be much more frequent than \href{https://cran.r-project.org/}{CRAN} releases.  So for up to date packages, as well as evolving information about the \pkg{pbd} project, see our website \href{https://pbdr.org/}{https://pbdr.org/}.


\subsection{Installation}
\label{sec:installation}

The \pkg{pbdBASE} package is available from the CRAN at
\url{https://cran.r-project.org}, and can be installed via a simple 
\begin{lstlisting}[language=rr,title=Installing pbdBASE]
install.packages("pbdBASE")
\end{lstlisting}
This assumes only that you have MPI installed and properly configured on your system.  If the user can successfully install the package's two principal dependencies, \pkg{pbdMPI}~\citep{Chen2012pbdMPIpackage} and \pkg{pbdSLAP}~\citep{Chen2012pbdSLAPpackage} (each available from the CRAN), then the installation for \pkg{pbdBASE} should go smoothly.  If you experience difficulty installing either these packages, you should see their documentation.


\subsection{Intended Audience}
\label{sec:more_examples}

The \pkg{pbdBASE} package is a dependency of \pkg{pbdDMAT}, and so anyone who wishes to use the latter package must first install \pkg{pbdBASE}.  However, much of the direct use of \pkg{pbdBASE} is intended only for extremely advanced users and developers.  A few exceptions are the \code{init.grid()} and \code{finalize()} functions, which will be outlined in the sections to follow.  The overwhelming majority of the remaining functions are either internal or for people deeply familiar with ScaLAPACK.





\subsection{Terminology}
Before beginning, we will make frequent use of concepts from the Single Program/Multiple Data (SPMD) paradigm.  If you are entirely unfamiliar with this approach to parallelism, or if you are unfamiliar with the \pkg{pbdMPI} package, then you are strongly encouraged to read the vignette~\citep{Chen2012pbdMPIvignette} contained in the \pkg{pbdMPI} package, as well as examine and digest its many examples in order to better understand what follows.

A concise explanation of SPMD is that it is an approach to parallel, distributed programming in which one program is written, and each processor runs that same program, though that program locally will often be interacting with different data.  This, in contrast to the manager/worker paradigm where one processor, the manager, is in charge of its workers, each of whom swear fealty to the manager.  So in SPMD, each processor believes itself to be the manager, the one in charge.  As a colleague, Dr. Russell Zaretzki put it, ``it's like academia.''
